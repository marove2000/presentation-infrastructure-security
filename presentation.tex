%\documentclass[10pt,aspectratio=169]{beamer}
\documentclass[10pt]{beamer}

\usetheme[progressbar=frametitle]{metropolis}
\usepackage{appendixnumberbeamer}

\usepackage{booktabs}
\usepackage[scale=2]{ccicons}

\usepackage{pgfplots}
\usepgfplotslibrary{dateplot}

\usepackage{xspace}
\newcommand{\themename}{\textbf{\textsc{metropolis}}\xspace}

\newcommand\blfootnote[1]{%
	\begingroup
	\renewcommand\thefootnote{}\footnote{#1}%
	\addtocounter{footnote}{-1}%
	\endgroup
}

\title{Sicherheit in der IT-Infrastruktur}
\subtitle{Beispiele aus der Praxis, Ausfallsicherheit, Backups, Verschlüsselung}
% \date{\today}
\date{14.03.2019}
\author{Timo Schindler}
\institute{OTH Regensburg}
% \titlegraphic{\hfill\includegraphics[height=1.5cm]{logo.pdf}}

\begin{document}

\maketitle

%%%%%%%%%%
%
% Inhalt
%
%%%%%%%%%%
\begin{frame}{Inhalt}
  \setbeamertemplate{section in toc}[sections numbered]
  \tableofcontents[hideallsubsections]
\end{frame}

\section{Einführung: Backup}

%%%%%%%%%%
%
% Über mich
%
%%%%%%%%%%
\begin{frame}[fragile]{Über Mich}

\begin{columns}[T,c,onlytextwidth]
\column{0.5\textwidth}
\begin{exampleblock}{Timo Schindler}
OTH Regensburg
\end{exampleblock}

\column{0.5\textwidth}
\begin{itemize}
  	\item Promotion: IT Security \& Machine Learning
  	\item Server- und Storage-Systeme
  	\item Virtualisierungs-Infrastruktur
  	\item Sysadmin mit Leidenschaft
\end{itemize}
\end{columns}
\end{frame}


%%%%%%%%%%
%
% Zentralisierung als Lösung
%
%%%%%%%%%%
\begin{frame}[fragile]{Zentralisierung als Lösung?}
\begin{columns}[T,c,onlytextwidth]
	\column{0.5\textwidth}
	\begin{itemize}
		\item In Zeiten von Cloud: Services wandern in die Rechenzentren
		\item Zentraler Zugriff für alle Benutzer
		\item Zentraler Schwachpunkt
		\item Vertrauen in Administratoren
		\item Sicherheit an Zentraler Stelle wichtiger den je!
	\end{itemize}

\column{0.5\textwidth}
 \begin{figure}
	\includegraphics[width=0.8\textwidth]{images/thereisnocloud-v2}
\end{figure}
\end{columns}
	\blfootnote{Bild: \href{https://fsfe.org}{fsfe.org}}
\end{frame}

%%%%%%%%%%
%
% Warum Backup?
%
%%%%%%%%%%
\begin{frame}[fragile]{Warum Backup?}
Gründe für Backups sehr divers. Datenverlust durch:
	\begin{itemize}
	\item Versehentliches Löschen
	\item Unberechtigte Veränderung durch Dritte
	\item Technischer Systemausfall
	\item Diebstahl, Sabotage, Betrug
	\item Katastrophen (Brand, Wasserschaden)
	\item Angriffe (z.B. Ransomware)
\end{itemize}
Backupmechanismen und -maßnahmen unterscheiden sich dadurch erheblich.
\end{frame}

%%%%%%%%%%
%
% Schutzziele der Informationssicherheit
%
%%%%%%%%%%
\begin{frame}[fragile]{Schutzziele der Informationssicherheit}
\begin{columns}[T,onlytextwidth]
	\column{0.45\textwidth}
	Allgemeine Schutzziele
	
	\metroset{block=fill}
	\begin{exampleblock}{Vertraulichkeit}
		Lesen nur durch autorisierte Benutzer
	\end{exampleblock}
	
	\begin{exampleblock}{Integrität}
		Keine unbemerkte Veränderung
	\end{exampleblock}
	
	\begin{exampleblock}{Verfügbarkeit}
		Verhinderung von Systemausfällen
	\end{exampleblock}
	
	\column{0.45\textwidth}
	Weiter Schutzziele
	
	\metroset{block=fill}
	\begin{exampleblock}{Authentizität}
		Echtheit bzw. Überprüfbarkeit eines Objektes
	\end{exampleblock}
	
	\begin{exampleblock}{Verbindlichkeit}
		Kein unzulässiges Abstreiten von Aktionen
	\end{exampleblock}
	
	\begin{exampleblock}{Zurechenbarkeit}
		Zuordnung einer Aktion auf Benutzer
	\end{exampleblock}
	
\end{columns}

Schutzziele können nur durch Zusammenspiel aus Hard- und Software erreicht werden.
\end{frame}

%%%%%%%%%%
%
% Maßstab für Ausfallsicherheit
%
%%%%%%%%%%
\begin{frame}[fragile]{Maßstab für Ausfallsicherheit}
\begin{alertblock}{Tier 1 - Die Holzklasse}
\end{alertblock}
		\begin{columns}[T,c,onlytextwidth]
		\column{0.6\textwidth}
			\begin{itemize}
			\item Keine Redundanz
			\item Jährliche Ausfallzeit 28,8 Stunden
			\item 99,67 \% Verfügbarkeit
			\item Wartung im Betrieb nicht möglich
			\item Nur ein Versorgungsweg für Kälte- und Energieverteilung
		\end{itemize}
		\column{0.4\textwidth}
	 \begin{figure}
		\includegraphics[width=1\textwidth]{images/tier1}
	\end{figure}
\end{columns}
\end{frame}

\begin{frame}[fragile]{Maßstab für Ausfallsicherheit}
\begin{alertblock}{Tier 2 - Einfache Redundanz im Rechenzentrum}
\end{alertblock}
\begin{columns}[T,c,onlytextwidth]
	\column{0.6\textwidth}
	\begin{itemize}
		\item Redundanz nur in Versorgungsweg
		\item Jährliche Ausfallzeit 22 Stunden
		\item 99,75 \% Verfügbarkeit
		\item Wartung im Betrieb bedingt möglich
		\item Redundanter Versorgungsweg für Kälte- und Energieverteilung
	\end{itemize}
	\column{0.4\textwidth}
	\begin{figure}
		\includegraphics[width=1\textwidth]{images/tier2}
	\end{figure}
\end{columns}
\end{frame}

\begin{frame}[fragile]{Maßstab für Ausfallsicherheit}
\begin{alertblock}{Tier 3 - Fehlertoleranz möglich}
\end{alertblock}
\begin{columns}[T,c,onlytextwidth]
	\column{0.6\textwidth}
	\begin{itemize}
		\item Redundanz in Versorgung
		\item Server mehrfach vorhanden
		\item Jährliche Ausfallzeit 1,6 Stunden
		\item 99,98 \% Verfügbarkeit
		\item Wartung im Betrieb möglich
		\item Redundanter Versorgungsweg für Kälte- und Energieverteilung
	\end{itemize}
	\column{0.4\textwidth}
	\begin{figure}
		\includegraphics[width=1\textwidth]{images/tier3}
	\end{figure}
\end{columns}
\end{frame}

\begin{frame}[fragile]{Maßstab für Ausfallsicherheit}
\begin{alertblock}{Tier 4 - Die Masterclass}
\end{alertblock}
\begin{columns}[T,c,onlytextwidth]
	\column{0.6\textwidth}
	\begin{itemize}
		\item Komplette doppelte Redundanz
		\item Server mehrfach vorhanden
		\item Jährliche Ausfallzeit 0,8 Stunden
		\item 99,991 \% Verfügbarkeit
		\item Wartung im Betrieb möglich
		\item Mehrfach redundanter Versorgungsweg für Kälte- und Energieverteilung
	\end{itemize}
	\column{0.4\textwidth}
	\begin{figure}
		\includegraphics[width=1\textwidth]{images/tier4}
	\end{figure}
\end{columns}
\end{frame}

%%%%%%%%%%
%
% RAID
%
%%%%%%%%%%
\begin{frame}[fragile]{RAID ist kein Backup!}
\begin{columns}[T,c,onlytextwidth]
	\column{0.6\textwidth}
	\begin{itemize}
		\item Daten werden auf mehrere Festplatten verteilt
		\item Relative Ausfallsicherheit von Festplatten
		\item Problem bei Systematischen Fehlern
		\item Problem bei bestimmten RAID-Leveln
		\item RAID ist unverzichtbar, aber kein Backup!
	\end{itemize}
	\column{0.4\textwidth}
	\begin{figure}
		\includegraphics[width=1\textwidth]{images/RAID_1}
	\end{figure}
\end{columns}
	\blfootnote{Bild: \href{https://de.wikipedia.org}{wikipedia.org}}
\end{frame}

%%%%%%%%%%
%
% RAID 6
%
%%%%%%%%%%
\begin{frame}[fragile]{RAID ja, aber welche Konfiguration?}
\begin{alertblock}{RAID 6 oder 60}
\end{alertblock}
\begin{columns}[T,c,onlytextwidth]
	\column{0.6\textwidth}
	\begin{itemize}
		\item Bis zu zwei Festplatten können ausfallen
		\item Bei der Wiederherstellung von Festplatten oft Ausfall weiterer Platte
		\item Gutes Preis/Leistungs-Verhältnis
	\end{itemize}
	\column{0.4\textwidth}
	\begin{figure}
		\includegraphics[width=1\textwidth]{images/RAID_6}
	\end{figure}
\end{columns}
\blfootnote{Bild: \href{https://de.wikipedia.org}{wikipedia.org}}
\end{frame}

%%%%%%%%%%
%
% RAID Z2
%
%%%%%%%%%%
\begin{frame}[fragile]{Noch besser: RAID Z2}
\begin{alertblock}{Zettabyte File System}
\end{alertblock}
\begin{columns}[T,c,onlytextwidth]
	\column{0.6\textwidth}
\begin{itemize}
	\item Spezielles Filesystem
	\item Als Software-RAID umgesetzt
	\item Ausfallsicherheit wie RAID 6
	\item Reparatur von Files durch Hashes
	\item Möglich: Deduplizierung \& Kompression
	\item Möglich: Verschlüsselung \& Caching
\end{itemize}
	\column{0.4\textwidth}
	\begin{figure}
		\includegraphics[width=1\textwidth]{images/zfs-self-healing}
	\end{figure}
\end{columns}

\blfootnote{Bild: \href{https://www.root.cz}{root.cz}}
\end{frame}

%%%%%%%%%%
%
% Signaturen und Hashing Algorithmen
%
%%%%%%%%%%
\begin{frame}[fragile]{Signaturen und Hashing Algorithmen}
\begin{columns}[T,c,onlytextwidth]
	\column{0.6\textwidth}
	\begin{itemize}
		\item Einwegfunktion
		\item Hash immer gleiche Größe
		\item Gleiche Datei erzeugt gleichen Hash
		\item Minimale Änderungen erzeugen völlig unterschiedlichen Hash
		\item Kryptographische Sicherheit
	\end{itemize}
	\column{0.4\textwidth}
	\begin{figure}
		\includegraphics[width=1\textwidth]{images/hashing}
	\end{figure}
\end{columns}
\end{frame}

%%%%%%%%%%
%
% Spezialfall: Revisionssichere Archivierung
%
%%%%%%%%%%
\begin{frame}[fragile]{Spezialfall: Revisionssichere Archivierung}
	\begin{itemize}
	\item Schutz vor Manipulation
	\item Schutz vor nachträglicher Änderung
	\item Wird oft durch kryptografische Signaturen sichergestellt
	\item Zertifizierte Systeme sehr teuer
	\item Nötig für Compliance, Finanz- und Gesundheitsdaten
\end{itemize}
\end{frame}

%%%%%%%%%%
%
% Verschlüsselung
%
%%%%%%%%%%
\begin{frame}[fragile]{Verschlüsselung}
	\begin{itemize}
	\item Backups: Beliebtes Ziel für Datenmanipulation und -diebstahl
	\item Backups oft nachlässige Sicherheit
	\item Verschlüsselung macht Backup unbequem
	\item Eigener Infrastruktur sollte nicht vertraut werden
	\item Transportverschlüsselung nicht vergessen
\end{itemize}
\end{frame}

%%%%%%%%%%
%
% Vertraue keinem Backup!
%
%%%%%%%%%%
\begin{frame}[fragile]{Vertraue keinem Backup!}
\begin{alertblock}{Niemals!}
\end{alertblock}
\begin{columns}[T,c,onlytextwidth]
	\column{0.6\textwidth}
	\begin{itemize}
		\item Backups prüfen
		\item Ernstfall simulieren
		\item Mehrstufige Backups
		\item Nochmal Backups prüfen!
	\end{itemize}
	\column{0.4\textwidth}
	\begin{figure}
		\includegraphics[width=1\textwidth]{images/failed-backup}
	\end{figure}
\end{columns}

\blfootnote{Bild: \href{https://mail.gnome.org/archives/deja-dup-list/2012-November/msg00000.html}{gnome.org}}
\end{frame}

%%%%%%%%%%
%
% Notfallplan
%
%%%%%%%%%%
\begin{frame}[fragile]{Notfallplan}
\begin{columns}[T,c,onlytextwidth]
	\column{0.6\textwidth}
	\begin{itemize}
		\item Ausfälle passieren...
		\item ...zu unmöglichsten Zeiten
		\item Notfallplan aufstellen
		\item Infrastruktur funktioniert nicht
		\item Dauer?
		\item Ist diese Zeit vertretbar?
	\end{itemize}
	\column{0.4\textwidth}
	\begin{figure}
		\includegraphics[width=1\textwidth]{images/dilbert}
	\end{figure}
\end{columns}

\blfootnote{Bild: \href{https://www.dilbert.com}{dilbert.com}}
\end{frame}

\section{Backupinfrastruktur in der Praxis}

\begin{frame}[fragile]{Struktur der OTH Regensburg}
\begin{alertblock}{Ausfälle}
	Ausfälle passieren
	Ausfälle passieren zu unmöglichsten Zeiten
	Ausfall von Systemadminstratoren?
	Was tun wenn etwas passiert?
\end{alertblock}
\end{frame}

\begin{frame}[fragile]{Problem: Redundante Klimatechnik}
\begin{alertblock}{Ausfälle}
	Ausfälle passieren
	Ausfälle passieren zu unmöglichsten Zeiten
	Ausfall von Systemadminstratoren?
	Was tun wenn etwas passiert?
\end{alertblock}
\end{frame}

\begin{frame}[fragile]{Problem: Stromausfälle}
\begin{alertblock}{Ausfälle}
	Ausfälle passieren
	Ausfälle passieren zu unmöglichsten Zeiten
	Ausfall von Systemadminstratoren?
	Was tun wenn etwas passiert?
\end{alertblock}
\end{frame}

\begin{frame}[fragile]{Problem: Angriffe}
\begin{alertblock}{Ausfälle}
	Ausfälle passieren
	Ausfälle passieren zu unmöglichsten Zeiten
	Ausfall von Systemadminstratoren?
	Was tun wenn etwas passiert?
\end{alertblock}
\end{frame}

\begin{frame}[fragile]{Problem: Netzwerkverteilung}
\begin{alertblock}{Ausfälle}
	Ausfälle passieren
	Ausfälle passieren zu unmöglichsten Zeiten
	Ausfall von Systemadminstratoren?
	Was tun wenn etwas passiert?
\end{alertblock}
\end{frame}

\begin{frame}[fragile]{Unbeabsichtigte Änderung/Löschung}
\begin{alertblock}{Ausfälle}
	An den User weitergeben
	
\end{alertblock}
\end{frame}

\begin{frame}[fragile]{Reaktionszeiten}
\begin{alertblock}{Ausfälle}
	Ausfälle passieren
	Ausfälle passieren zu unmöglichsten Zeiten
	Ausfall von Systemadminstratoren?
	Was tun wenn etwas passiert?
\end{alertblock}
\end{frame}

\section{Grundlagen: Verschlüsselung und Signierung}

\subsection{Grundlagen}

\begin{frame}[fragile]{Warum Verschlüsselung?}
\begin{alertblock}{Ausfälle}
Nötig in der Praxis?
Vertrauen in den Admin
Vertrauen in die Infrastruktur
\end{alertblock}
\end{frame}

\begin{frame}[fragile]{Symmetrische Verschlüsselung}
\begin{alertblock}{Ausfälle}
	Ausfälle passieren
	Ausfälle passieren zu unmöglichsten Zeiten
	Ausfall von Systemadminstratoren?
	Was tun wenn etwas passiert?
\end{alertblock}
\end{frame}

\begin{frame}[fragile]{Asymmetrische Verschlüsselung}
\begin{alertblock}{Ausfälle}
	Ausfälle passieren
	Ausfälle passieren zu unmöglichsten Zeiten
	Ausfall von Systemadminstratoren?
	Was tun wenn etwas passiert?
\end{alertblock}
\end{frame}

\begin{frame}[fragile]{Transportverschlüsselung}
\begin{alertblock}{Ausfälle}
	Ausfälle passieren
	Ausfälle passieren zu unmöglichsten Zeiten
	Ausfall von Systemadminstratoren?
	Was tun wenn etwas passiert?
\end{alertblock}
\end{frame}

\begin{frame}[fragile]{Festplattenverschlüsselung}
\begin{alertblock}{Ausfälle}
	Ausfälle passieren
	Ausfälle passieren zu unmöglichsten Zeiten
	Ausfall von Systemadminstratoren?
	Was tun wenn etwas passiert?
\end{alertblock}
\end{frame}

\begin{frame}[fragile]{Signaturalgorithmen}
\begin{alertblock}{Ausfälle}
Server von DFN beschreiben
Warum brauchen wir das?
\end{alertblock}
\end{frame}

\begin{frame}[fragile]{PGP und S/MIME}
\begin{alertblock}{Ausfälle}
	Server von DFN beschreiben
	Warum brauchen wir das?
\end{alertblock}
\end{frame}

\section{Verschlüsselung und Signierung in der Praxis}

\begin{frame}[fragile]{https TLS}
\begin{alertblock}{Ausfälle}
	Ausfälle passieren
	Ausfälle passieren zu unmöglichsten Zeiten
	Ausfall von Systemadminstratoren?
	Was tun wenn etwas passiert?
\end{alertblock}
\end{frame}

\begin{frame}[fragile]{Signatur von Backups}
\begin{alertblock}{Ausfälle}
	Ausfälle passieren
	Ausfälle passieren zu unmöglichsten Zeiten
	Ausfall von Systemadminstratoren?
	Was tun wenn etwas passiert?
\end{alertblock}
\end{frame}

\begin{frame}[fragile]{Verschlüsselung von Dateien und Backups}
\begin{alertblock}{Ausfälle}
	Ausfälle passieren
	Ausfälle passieren zu unmöglichsten Zeiten
	Ausfall von Systemadminstratoren?
	Was tun wenn etwas passiert?
\end{alertblock}
\end{frame}

\begin{frame}[fragile]{Signaturalgorithmen}
\begin{alertblock}{Ausfälle}
	Ausfälle passieren
	Ausfälle passieren zu unmöglichsten Zeiten
	Ausfall von Systemadminstratoren?
	Was tun wenn etwas passiert?
\end{alertblock}
\end{frame}

\begin{frame}[fragile]{Email-Verschlüsselung in der Praxis}
\begin{alertblock}{Ausfälle}
	Ausfälle passieren
	Ausfälle passieren zu unmöglichsten Zeiten
	Ausfall von Systemadminstratoren?
	Was tun wenn etwas passiert?
\end{alertblock}
\end{frame}

\begin{frame}[fragile]{Ausflug: Whatsapp und Co}
\begin{alertblock}{Ausfälle}
	Ausfälle passieren
	Ausfälle passieren zu unmöglichsten Zeiten
	Ausfall von Systemadminstratoren?
	Was tun wenn etwas passiert?
\end{alertblock}
\end{frame}

\begin{frame}[fragile]{VPN}
\begin{alertblock}{Ausfälle}
	Ausfälle passieren
	Ausfälle passieren zu unmöglichsten Zeiten
	Ausfall von Systemadminstratoren?
	Was tun wenn etwas passiert?
\end{alertblock}
\end{frame}

\section{Zusammenfassung}

\begin{frame}[fragile]{Zusammenfassung}
\begin{alertblock}{Ausfälle}
	Ausfälle passieren
	Ausfälle passieren zu unmöglichsten Zeiten
	Ausfall von Systemadminstratoren?
	Was tun wenn etwas passiert?
\end{alertblock}
\end{frame}



\begin{frame}{Metropolis titleformats}
	\themename supports 4 different titleformats:
	\begin{itemize}
		\item Regular
		\item \textsc{Smallcaps}
		\item \textsc{allsmallcaps}
		\item ALLCAPS
	\end{itemize}
	They can either be set at once for every title type or individually.
\end{frame}

{
    \metroset{titleformat frame=smallcaps}
\begin{frame}{Small caps}
	This frame uses the \texttt{smallcaps} titleformat.

	\begin{alertblock}{Potential Problems}
		Be aware, that not every font supports small caps. If for example you typeset your presentation with pdfTeX and the Computer Modern Sans Serif font, every text in smallcaps will be typeset with the Computer Modern Serif font instead.
	\end{alertblock}
\end{frame}
}

{
\metroset{titleformat frame=allsmallcaps}
\begin{frame}{All small caps}
	This frame uses the \texttt{allsmallcaps} titleformat.

	\begin{alertblock}{Potential problems}
		As this titleformat also uses smallcaps you face the same problems as with the \texttt{smallcaps} titleformat. Additionally this format can cause some other problems. Please refer to the documentation if you consider using it.

		As a rule of thumb: Just use it for plaintext-only titles.
	\end{alertblock}
\end{frame}
}

{
\metroset{titleformat frame=allcaps}
\begin{frame}{All caps}
	This frame uses the \texttt{allcaps} titleformat.

	\begin{alertblock}{Potential Problems}
		This titleformat is not as problematic as the \texttt{allsmallcaps} format, but basically suffers from the same deficiencies. So please have a look at the documentation if you want to use it.
	\end{alertblock}
\end{frame}
}

\section{Elements}

\begin{frame}[fragile]{Typography}
      \begin{verbatim}The theme provides sensible defaults to
\emph{emphasize} text, \alert{accent} parts
or show \textbf{bold} results.\end{verbatim}

  \begin{center}becomes\end{center}

  The theme provides sensible defaults to \emph{emphasize} text,
  \alert{accent} parts or show \textbf{bold} results.
\end{frame}

\begin{frame}{Font feature test}
  \begin{itemize}
    \item Regular
    \item \textit{Italic}
    \item \textsc{SmallCaps}
    \item \textbf{Bold}
    \item \textbf{\textit{Bold Italic}}
    \item \textbf{\textsc{Bold SmallCaps}}
    \item \texttt{Monospace}
    \item \texttt{\textit{Monospace Italic}}
    \item \texttt{\textbf{Monospace Bold}}
    \item \texttt{\textbf{\textit{Monospace Bold Italic}}}
  \end{itemize}
\end{frame}

\begin{frame}{Lists}
  \begin{columns}[T,onlytextwidth]
    \column{0.33\textwidth}
      Items
      \begin{itemize}
        \item Milk \item Eggs \item Potatos
      \end{itemize}

    \column{0.33\textwidth}
      Enumerations
      \begin{enumerate}
        \item First, \item Second and \item Last.
      \end{enumerate}

    \column{0.33\textwidth}
      Descriptions
      \begin{description}
        \item[PowerPoint] Meeh. \item[Beamer] Yeeeha.
      \end{description}
  \end{columns}
\end{frame}
\begin{frame}{Animation}
  \begin{itemize}[<+- | alert@+>]
    \item \alert<4>{This is\only<4>{ really} important}
    \item Now this
    \item And now this
  \end{itemize}
\end{frame}
\begin{frame}{Figures}
  \begin{figure}
    \newcounter{density}
    \setcounter{density}{20}
    \begin{tikzpicture}
      \def\couleur{alerted text.fg}
      \path[coordinate] (0,0)  coordinate(A)
                  ++( 90:5cm) coordinate(B)
                  ++(0:5cm) coordinate(C)
                  ++(-90:5cm) coordinate(D);
      \draw[fill=\couleur!\thedensity] (A) -- (B) -- (C) --(D) -- cycle;
      \foreach \x in {1,...,40}{%
          \pgfmathsetcounter{density}{\thedensity+20}
          \setcounter{density}{\thedensity}
          \path[coordinate] coordinate(X) at (A){};
          \path[coordinate] (A) -- (B) coordinate[pos=.10](A)
                              -- (C) coordinate[pos=.10](B)
                              -- (D) coordinate[pos=.10](C)
                              -- (X) coordinate[pos=.10](D);
          \draw[fill=\couleur!\thedensity] (A)--(B)--(C)-- (D) -- cycle;
      }
    \end{tikzpicture}
    \caption{Rotated square from
    \href{http://www.texample.net/tikz/examples/rotated-polygons/}{texample.net}.}
  \end{figure}
\end{frame}
\begin{frame}{Tables}
  \begin{table}
    \caption{Largest cities in the world (source: Wikipedia)}
    \begin{tabular}{lr}
      \toprule
      City & Population\\
      \midrule
      Mexico City & 20,116,842\\
      Shanghai & 19,210,000\\
      Peking & 15,796,450\\
      Istanbul & 14,160,467\\
      \bottomrule
    \end{tabular}
  \end{table}
\end{frame}
\begin{frame}{Blocks}
  Three different block environments are pre-defined and may be styled with an
  optional background color.

  \begin{columns}[T,onlytextwidth]
    \column{0.5\textwidth}
      \begin{block}{Default}
        Block content.
      \end{block}

      \begin{alertblock}{Alert}
        Block content.
      \end{alertblock}

      \begin{exampleblock}{Example}
        Block content.
      \end{exampleblock}

    \column{0.5\textwidth}

      \metroset{block=fill}

      \begin{block}{Default}
        Block content.
      \end{block}

      \begin{alertblock}{Alert}
        Block content.
      \end{alertblock}

      \begin{exampleblock}{Example}
        Block content.
      \end{exampleblock}

  \end{columns}
\end{frame}
\begin{frame}{Math}
  \begin{equation*}
    e = \lim_{n\to \infty} \left(1 + \frac{1}{n}\right)^n
  \end{equation*}
\end{frame}
\begin{frame}{Line plots}
  \begin{figure}
    \begin{tikzpicture}
      \begin{axis}[
        mlineplot,
        width=0.9\textwidth,
        height=6cm,
      ]

        \addplot {sin(deg(x))};
        \addplot+[samples=100] {sin(deg(2*x))};

      \end{axis}
    \end{tikzpicture}
  \end{figure}
\end{frame}
\begin{frame}{Bar charts}
  \begin{figure}
    \begin{tikzpicture}
      \begin{axis}[
        mbarplot,
        xlabel={Foo},
        ylabel={Bar},
        width=0.9\textwidth,
        height=6cm,
      ]

      \addplot plot coordinates {(1, 20) (2, 25) (3, 22.4) (4, 12.4)};
      \addplot plot coordinates {(1, 18) (2, 24) (3, 23.5) (4, 13.2)};
      \addplot plot coordinates {(1, 10) (2, 19) (3, 25) (4, 15.2)};

      \legend{lorem, ipsum, dolor}

      \end{axis}
    \end{tikzpicture}
  \end{figure}
\end{frame}
\begin{frame}{Quotes}
  \begin{quote}
    Veni, Vidi, Vici
  \end{quote}
\end{frame}

{%
\setbeamertemplate{frame footer}{My custom footer}
\begin{frame}[fragile]{Frame footer}
    \themename defines a custom beamer template to add a text to the footer. It can be set via
    \begin{verbatim}\setbeamertemplate{frame footer}{My custom footer}\end{verbatim}
\end{frame}
}

\begin{frame}{References}
  Some references to showcase [allowframebreaks] \cite{knuth92,ConcreteMath,Simpson,Er01,greenwade93}
\end{frame}

\section{Conclusion}

\begin{frame}{Summary}

  Get the source of this theme and the demo presentation from

  \begin{center}\url{github.com/matze/mtheme}\end{center}

  The theme \emph{itself} is licensed under a
  \href{http://creativecommons.org/licenses/by-sa/4.0/}{Creative Commons
  Attribution-ShareAlike 4.0 International License}.

  \begin{center}\ccbysa\end{center}

\end{frame}

{\setbeamercolor{palette primary}{fg=black, bg=yellow}
\begin{frame}[standout]
  Questions?
\end{frame}
}

\appendix

\begin{frame}[fragile]{Backup slides}
  Sometimes, it is useful to add slides at the end of your presentation to
  refer to during audience questions.

  The best way to do this is to include the \verb|appendixnumberbeamer|
  package in your preamble and call \verb|\appendix| before your backup slides.

  \themename will automatically turn off slide numbering and progress bars for
  slides in the appendix.
\end{frame}

\begin{frame}[allowframebreaks]{References}

  \bibliography{presentation}
  \bibliographystyle{abbrv}

\end{frame}

\end{document}
